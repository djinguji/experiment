\documentclass[12pt]{article}

\usepackage{fullpage}
\usepackage{amsmath}
\usepackage{amssymb}
\usepackage{mathrsfs}
\usepackage{graphics}
%\usepackage{url}
%\usepackage{lingmacros}
%\usepackage{tipa}
%\usepackage{stmaryrd}
%\usepackage{devanagari}

\begin{document}

\title{AMath 503 \\ Homework 5}
\author{Dan Jinguji \\ 7339426}
\date{Due: Monday 31 May 2016}

\maketitle

\renewcommand\thesection {\arabic{section}.}
\renewcommand\thesubsection {(\alph{subsection})}
\renewcommand\thesubsubsection{\alph{subsubsection}.}
\newcommand{\pderiv}[1]{\ensuremath{\frac{\partial}{\partial #1}}}
\newcommand{\deriv}[1]{\ensuremath{\frac{d}{d #1}}}
\newcommand{\psec}[1]{\ensuremath{\frac{\partial^2}{\partial #1^2}}}
\newcommand{\pr}{\ensuremath{^{\prime}}}
\newcommand{\dpr}{\ensuremath{^{\prime\prime}}}
%\newcommand{\and}{\,\&\;}
\newcommand{\llb}{\ensuremath{\left [}}
\newcommand{\rrb}{\ensuremath{\right ]}}
\newcommand{\llp}{\ensuremath{\left (}}
\newcommand{\rrp}{\ensuremath{\right )}}
\newcommand{\llg}{\ensuremath{\left \{}}
\newcommand{\rrg}{\ensuremath{\right .}}
\newcommand{\abs}[1]{\ensuremath{\lvert #1 \rvert}}
\newcommand{\eval}[3]{\ensuremath{\left. #1 \right\rvert_{#2}^{#3}}}
\newcommand{\at}{\ensuremath{\,\mathrm{at}\,}}
%\newcommand{\blt}{\ensuremath{\bullet\;}}
%\newcommand{\sem}[1]{\ensuremath{\llbracket\mathrm{#1}\rrbracket}}
%\newcommand{\und}{\ensuremath{\!\_\,}}
%\newcommand{\und}{\ensuremath{\_\,}}
\newcommand{\goto}{\ensuremath{\rightarrow}}
\newcommand{\mathword}[1]{\ensuremath{\; \mathrm{#1} \;}}
%\newcommand{\dom}[1]{\ensuremath{\mathbf{D}_{\mathrm{#1}}}}
%\newcommand{\lamb}[3]{{[\ensuremath{\lambda\mathrm{#1}\;\in\;{#2}\;.\;}{#3}]}}
%\newcommand{\ip}[1]{\textipa{#1}}
%\newcommand{\rd}{\textrtaild}
%\newcommand{\rr}{\textrtailr}
%\newcommand{\rs}{\textrtails}
%\newcommand{\rt}{\textrtailt}
\newcommand{\ueq}{\ensuremath{u_\mathrm{eq}}}
\newcommand{\utr}{\ensuremath{u_\mathrm{trans}}}

\section{Solving a Nonhomogeneous System}

Consider the following nonhomogeneous system:
\begin{align*}
&\mathrm{PDE}: \psec{t}u = \psec{x}u + 1,\;0 < x < 1 \\
&\mathrm{BC}: u(0,t) = 0 = u(1,t) \\
&\mathrm{IC}: u(x,0) = 0, \;\pderiv{t}u(x,0) = 0
\end{align*}
Solve this problem in two ways.


\subsection{By eigenfunction expansion}

By eigenfunction expansion, that is, expand the solution in the form of an infinite sum of eigenfunctions (in space) of the homogeneous system with unknown coefficient (which is a
function of time) in front of each eigenfunction. Do the same for the forcing term, ``1''. Then solve an ODE in time.
\[ \]
We start by solving the homogeneous PDE using an eigenfunction expansion meets the boundary conditions.
\begin{align*}
\psec{t}u &= \psec{x}u \\
\mathword{Let} u(x,t) &= T(t)X(x) \\
T\dpr X &= X\dpr T \\
\frac{T\dpr}{T} &= \frac{X\dpr}{X} = \lambda^2
\end{align*}
To meet the boundary conditions, $u(0,t) = u(1,t) = 0$, we choose the eigenfunction expansion as a sine series, $\lambda_n = n\pi$.
\[ u(x,t) = \sum_{n=1}^\infty b_n(t)\sin(n\pi x) \]
Now, we express the forcing function in terms of this same eigenfunction expansion.
\[ 1 = \sum_{n=1}^\infty f_n(t)\sin(n\pi x) \]
Substituting back into the original PDE,
\[ \sum_{n=1}^\infty b_n\dpr(t)\sin(n\pi x) = \sum_{n=1}^\infty -(n\pi)^2b_n(t) \sin(n\pi x) + \sum_{n=1}^\infty f_n(t)\sin(n\pi x) \]
Based on the orthogonality of the sine series,
\begin{align*}
b_n\dpr(t) &= f_n(t) - (n\pi)^2b_n(t) \\
b_n\dpr(t) + (n\pi)^2b_n(t) &= f_n(t)
\end{align*}
Solving this second-order ODE using variation of parameters gives:
\begin{align*} b_n(t) &= c_{1,n}\cos(n\pi t) + c_{2,n}\sin(n\pi t) + \\
&\quad \sin(n\pi x)\int_1^x\frac{f_n(\xi)\cos(n\pi\xi)}{n\pi}d\xi + \cos(n\pi x)\int_1^x -\frac{f_n(\xi)\sin(n\pi\xi)}{n\pi}d\xi
\end{align*}
From our solution to Homework 2, also derivable from Equation (4.6) in the notes:
\[
f_n(t) = \begin{array}{cl}
0 & n,\mathword{even} \\
\frac{4}{n\pi} & n,\mathword{odd}
\end{array}
\]

\newpage
\subsection{By steady state}

By first finding the steady state solution to the nonhomogeneous equation and then the transient solution; the latter is the difference between the true solution and the steady state solution and should satisfy a homogeneous equation.
\[ \]
Since the forcing function for this PDE does not depend on time, the solution can be written as the sum of an equilibrium term and a transient term.
\[ u(x,t) = u_{\mathrm{eq}}(x) + u_{\mathrm{trans}}(x,t) \]

Considering the original PDE as $t \goto \infty$, we have:
\[ \psec{x}\ueq = -1 \]
Integrating twice, we get:
\[ \ueq(x) = -\frac{x^2}{2} \]

Substituting this back into the original PDE we have:
\[ \psec{t}\utr = \psec{x}\utr, \; 0 < x < 1 \]
with the same boundary conditions, $\utr(0,t) = \utr(1,t) = 0$.

Solving this homogeneous PDE, we have
\begin{align*}
\utr(x,t) &= T(t)X(t) \\
\frac{T\dpr}{T} &= \frac{X\dpr}{X} = \lambda^2 \\
\end{align*}
To meet the boundary conditions, use sine series.
\[ X(x) = \sum_{n=1}^\infty \sin(n\pi x) \]
\begin{align*}
T\dpr &= \lambda^2 T \\
T(t) &= \sum_{n=1}^\infty A_n\sin(n\pi t) + B_n\cos(n\pi t) \\
\utr(x,t) &= \sum_{n=1}^\infty \llp A_n\sin(n\pi t) + B_n\cos(n\pi t)\rrp \sin(n\pi x)
\end{align*}
However the initial conditions have changed to account for the steady-state solution.
\begin{align*}
\pderiv{t}\utr &= 0 \\
\pderiv{t}\utr(x,0) &= \sum_{n=1}^\infty n\pi A_n\sin(n\pi x) = 0 \\
\mathword{So,} A_n &= 0 \\
\utr(x, 0) &= u(x, 0) - \ueq(x) \\
&= 0 + \frac{x^2}{2} = \frac{x^2}{2} \\
\utr(x,0) &= \sum_{n=1}^\infty B_n\sin(n\pi x) = \frac{x^2}{2} \\
A_n &= 2\int_0^1 \frac{x^2}{2}\sin(n\pi x)dx \\
&= \frac{(2 - \pi^2n^2)\cos(n\pi)+2n\pi\sin(n\pi) - 2}{\pi^3n^3}\\
\end{align*}

\newpage
\section{Nonhomogeneous Wave Equation}
Consider the following one-dimension nonhomogeneous wave equation (assume the boundary conditions are such that the solution is integrable):
\begin{equation*}
\mathrm{PDE:} \llp\psec{t} - c^2\psec{x}\rrp u = \delta(x - \xi)\delta(t - \tau), \;t > 0, \; -\infty>x>\infty, \;-\infty<\xi<\infty
\end{equation*}
subject to the zero initial conditions:
\begin{equation*}
u = 0 \mathword{and} \pderiv{t}u = 0, \mathword{at} t = 0.
\end{equation*}
For the solution to be integrable, the function $u \goto 0$ as $x \goto \pm\infty$.

\subsection{Equivalence to Homogeneous System}
\label{part1}

Show that the above problem is the same as the following homogeneous problem:
\begin{equation*}
\mathrm{PDE}: \llp\psec{t} - c^2\psec{x}\rrp u = 0, \;t > \tau.
\end{equation*}
subject to the following ``initial condition'' at $t = \tau$:
\
\begin{equation*}
u = 0 \mathword{at} t=\tau, \mathword{and} \pderiv{t}u = \delta(x - \xi) \mathword{at} t = \tau.
\end{equation*}
And $u \equiv 0$ for $t < \tau$.
\[ \]
The $\delta(t - \tau)$ factor in the ``forcing'' term means that we could consider the problem as broken into two segments: $t < \tau$ and $t \ge \tau$.

Given the initial conditions for the original problem, $u(x,0) = 0$ and $\pderiv{t}u(x,0) = 0$. The forcing term will have no effect until $t = \tau$, so we can characterize the original PDE as $u(x,t) = 0$ for $t < \tau$. There is nothing to perturb the state until the ``pulse'' at $t = \tau$.

For the solution $u$ to be integrable, $u$ must be continuous at $t = \tau$, as well as at $x = \xi$. There can be a ``jump'' in the derivative at this point, which would result in the second derivative being the $\delta$ function, as seen in the original PDE. Since the solution $u$ is continuous at $t = \tau$, $u(x,\tau) = 0$, since $u(x,t) = 0$ for $t < \tau$, as shown above.

Considering $t > \tau$, the forcing function $\delta(x - \xi)\delta(t - \tau)$ is zero, by definition of the $\delta$ function. So, considering the partial time domain, $t > \tau$, the value of the PDE would be zero, since the $\delta(t - \tau)$ term of the forcing function is zero by definition.
\[ \llp\psec{t} - c^2\psec{x}\rrp u = 0, \;t > \tau \]

The only thing that must be resolved now is the rest of the ``initial condition'' for this partial time domain. $\pderiv{t}u(x,\tau) = \delta(x - \xi)$, localizing the pulse to $x = \xi$ at $t = \tau$. We have ``broken'' the continuity of $t$ by splitting the time domain into these two segments. So, to capture the perturbation at $t = \tau$, we use the $\delta$ function at $x = \xi$.

\subsection{Solution to \ref{part1}}
\label{part2}

Solve the problem defined by \ref{part1} either  by Fourier transform or by D'Alembert's method.
\[ \]
By D'Alembert, 
\[ \llp\pderiv{t} - c\pderiv{x}\rrp\llp\pderiv{t} + c\pderiv{x}\rrp = 0 \] 
This is solvable as $L(x - ct)$ and $R(x + ct)$.
This leads to a general solution, 
\[ u(x,t) = L(x + ct) + R(x - ct) \]
So, 
\begin{align*}
L(x) + R(x) &= 0 \\
L(x) - R(x) &= \frac{1}{c}\int^x \delta(x\pr-\xi)dx\pr + K \\
&= \frac{1}{c} \\
2L(x) &= \frac{1}{c} \\
L(x) &= \frac{1}{2c} \\
2R(x) &= -\frac{1}{c} \\
R(x) &= -\frac{1}{2c}
\end{align*}

Assume a solution:
\[ u(x,t_\tau) = \frac{1}{2\pi}\int_{-\infty}^\infty U(\omega,t_\tau)e^{-i\omega x}d\omega, \mathword{where} t_\tau = t - \tau \]
Then, the PDE in \ref{part1} becomes:
\[ \frac{1}{2\pi}\int_{-\infty}^\infty\llp U_{tt}(\omega,t_\tau) + c^2\omega^2U(\omega,t_\tau)\rrp e^{-i\omega x}d\omega = 0 \]
So, 
\[U_{tt} + c^2\omega^2 U = 0 \]
This has the general solution:
\[ U(\omega,t) = A(\omega)\sin(c\omega t) + B(\omega)\cos(c\omega t) \]
Applying the ``initial conditions'', $u(x,\tau) = 0$ and $\pderiv{t}u(x,\tau) = \delta(x-\xi)$, we get $U_t(\omega,\tau) = 1$ and $U(\omega,\tau) = 0$.

Since $U(\omega,\tau) = 0$, $B(\omega) = 0$. $U_t(\omega,\tau) = c\omega A(\omega)\cos(c\omega\tau) = 1$. So, $A(\omega) = \frac{1}{c\omega}$.

\subsection{Application to Nonhomogeneous}

Use the result in \ref{part2} to solve:
\begin{align*}
&\mathrm{PDE}: \psec{t}u -c^2\psec{x}u = Q(x,t), \; -\infty < x < \infty, \; t > 0 \\
&\mathrm{BC}: u(x,t) \goto 0 \mathword{as} x \goto \pm\infty, \; t > 0 \\
&\mathrm{IC}: u(x,0) = 0, \;\pderiv{t}u(x,0) = 0, \; -\infty < x < \infty,
\end{align*}
where the forcing term $Q$ is given by:
\[
Q(x,t) = \llg \begin{array}{ll}
 1 & \mathword{for} -10<x<10,\; t>0 \\
 0 & \mathword{otherwise}
\end{array} \rrg \]


\end{document}
