\documentclass[12pt]{article}

\usepackage{fullpage}
\usepackage{amsmath}
\usepackage{graphics}
\usepackage{url}
\usepackage{lingmacros}
\usepackage{tipa}
\usepackage{stmaryrd}
\usepackage{devanagari}

\begin{document}

\title{Ling 454 \\ Homework 5}
\author{Dan Jinguji \\ 7339426}
\date{Due: Thursday 8 December 2011}

\maketitle

\renewcommand\thesection {\arabic{section}:}
\renewcommand\thesubsection {\arabic{subsection}.}
\renewcommand\thesubsubsection{\alph{subsubsection}.}
\newcommand{\and}{\,\&\;}
\newcommand{\llb}{\ensuremath{\llbracket}}
\newcommand{\rrb}{\ensuremath{\rrbracket}}
\newcommand{\blt}{\ensuremath{\bullet\;}}
\newcommand{\sem}[1]{\ensuremath{\llbracket\mathrm{#1}\rrbracket}}
%\newcommand{\und}{\ensuremath{\!\_\,}}
\newcommand{\und}{\ensuremath{\_\,}}
\newcommand{\then}{\ensuremath{\rightarrow}}
\newcommand{\dom}[1]{\ensuremath{\mathbf{D}_{\mathrm{#1}}}}
\newcommand{\lamb}[3]{{[\ensuremath{\lambda\mathrm{#1}\;\in\;{#2}\;.\;}{#3}]}}
\newcommand{\ip}[1]{\textipa{#1}}
%\newcommand{\rd}{\textrtaild}
%\newcommand{\rr}{\textrtailr}
%\newcommand{\rs}{\textrtails}
%\newcommand{\rt}{\textrtailt}

\section{Exercise 9.3: Semantic change}

Look up the following words in a dictionary which provides basic etymologies for words. ({\it The Oxford English Dictionary} is generally recognized as the primary authority in this area and is recommended here, although a number of other dictionaries also provide useful etymological information.) Determine what change in meaning has taken place in each word. State which type of semantic change is involved (from among the types defined in this chapter).

For example, if you were to see {\it villian} in the list, you would look it up and find out that it originally meant `person of the villa/farm' but has changed its meaning to `criminal, scoundrel', and you would state that this is an example of {\it denegration} (or {\it pejoration}).

\subsection{corpse}

Common current meaning: `dead body, particularly of a human' $<$ French {\it cors} and variant {\it corps}, originally, Latin {\it corpus} `body'.

This is an example of narrowing, moving from denoting any body to denoting a dead human body.

It is interesting to note that there is an alternate English word {\it corse} with the same etymology which means `living body'.

\subsection{crafty}

Common current meaning: `skillful in devising and carrying out underhand or evil schemes'. Earlier meaning (now archaic): `strong, powerful, dexterous'.

This is an example of pejoration. The original meaning was based on this word being the adjectival form of {\it craft} `power, skill'. The notion has become one of malevolence, possibly through association with terms such as `witchcraft'.

\subsection{disease}

Common current meaning: `a species of disorder or ailment'. Earlier meaning: `absence of ease, disquiet', by extension `a cause of discomfort or distress'.

This is an example of metonymy and narrowing. The initial meaning is clear from the component parts of the word, {\it dis-} `contra, against' {\it ease}. This was extended through metonymy to the source of the distress or discomfort. And then to one specific origin of distress or discomfort, a pathological (somatic) condition.

\subsection{fame}

Common current meaning: `the character attributed to a person \ldots by report or generally entertained, reputation; usually in good sense'. Earlier meaning: `that which people say, report, rumor'.

This is an example of amelioration. The earlier meaning was a generic term for `that which is said about some object'. This narrowed so that it generally applied only to good report. Hence, the term changed from being neutral to positive, amelioration.

\subsection{journey}

Common current meaning: `travel, one which has a beginning and end in time or place, particularly when viewed as a whole'. Original meaning: `a day's time' $<$ Old French {\it jornee} `a day's time'. Intermediate meaning: `a day's travel; the distance traveled in a day'.

This is an example of metonymy and synecdoche. The transfer of meaning from `a day's time' to `the distance traveled in a day' is metonymy. Then, the transfer from `one day's distance' to `the complete trek' is synecdoche.

\subsection{officious}

Common current meaning: `pompous or self-asserting in asserting one's authority'. Earlier meaning: `performing the requirements of one's office'. Intermediate meaning: `eager to serve or help', now rare.

The is an example of pejoration. As with many terms, the original meaning can be found in the component parts, pertaining to one's office, specifically to the appropriate discharge of the obligations of the office, hence `dutiful'. This then transferred to an attitudinal assessment of the officeholder as one obliging or eager to serve. The term underwent pejoration to denote unduly forward in offering help, interfering, obtrusive, and eventually to self-important and pompous.

\subsection{science}

Common current meaning: `a particular branch of learning or study, especially used in contrast to {\it art}'. Earlier meaning: `knowledge' $<$ Latin {\it sc\=ire} `to know'.

This is an example of narrowing. The term originally meant that which is known, i.e. `knowledge'. It has narrowed to mean those things which are known empirically through the application of previous knowledge and conscious application of principles. This is in distinction to {\it art} which has more of a connotation of knowledge of methods to achieve some result.

\subsection{starve}

Common current meaning: `to die from lack of food', now, by hyperbole `to be extremely hungry', potentially metaphorically, as in {\it starved for attention}.
Earlier meaning: `to die'.

This is an example of narrowing, with the original meaning of `to die' being narrowed such that the term now indicates the cause of death. This term has also undergone hyperbole, as noted above.

\subsection{vulgar}

Common current meaning: `having a common and offensively mean character, lacking refinement or good taste, uncultured'. Original meaning: `common' $<$ Latin {\it vulg\=aris} `of the common people'.

This is an example of pejoration. This was originally an adjective indicating `of or concerning the common people'. The distinction was then made between the {\it elite} and the {\it common}, with the {\it common} now indicating the base, the uneducated, the unrefined. Thus the term focused on the negative aspects of `the common folk', the `unwashed masses', if you will.

\section{Exercise 9.4}

In the following examples of semantic change, identify the kind of semantic change involved (widening, narrowing, metonymy,and so on.)

\addtocounter{subsection}{+1}
\subsection{}

Spanish {\it dinero} `money' $<$ Latin {\it d\=en\=arius} `coin (of a particular denomination)'.

This is an example of widening, moving from a specific value of coin to money in general.

\addtocounter{subsection}{+1}
\subsection{}

Spanish {\it segar} ` to reap (to cut grain, grass with a scythe)' $<$ Latin {\it sec\=are} `to cut'.

This is an example of narrowing, moving from cutting, in general, to a specific type of cutting, especially the harvest of grain or grass with a scythe.

\addtocounter{subsection}{+1}
\subsection{}

Mexican Spanish {\it muchacha}, formerly only `girl', now has a primary meaning `maid, servant woman' in some contexts.

This is an example of narrowing. The original denotation of the term was any young female person. It has been primarily limited in Mexican Spanish to refer to young women who are domestic servants. A similar narrowing has occurred with the English gloss, the word `maid', the original sense preserved now in the form `maiden'. It is not clear if this term may also have undergone some pejoration. The analogous English term `boy' has, in certain contexts, experienced a marked degradation in its referent.

\addtocounter{subsection}{+1}
\subsection{}

English {\it gay} `homosexual' is the result of a recent semantic shift, where the original sense, `cheerful, lively', has become secondary; the shift to the `homosexual' sense perhaps came through other senses, `given to social pleasures, licentious', which the word had.

This is an example of metonymy. The apparent association is between the homosexual lifestyle and licentiousness or being given to social pleasures. I believe there was also a later, conscious choice by the social activist segment of the homosexual community to promote `gay' as the preferred term. Perhaps because there was already some usage of the term and the then-primary denotation as `cheerful, lively' was viewed as desirable within the community which had generally been suppressed and down-trodden.

\addtocounter{subsection}{+1}
\subsection{}

French {\it cuisse} `thigh' $<$ Latin {\it coxa} `hip' (Spanish {\it cojo} `lame, crippled' is thought also to be from Latin {\it coxa} `hip').

This is an example of metonymy. The association comes through the anatomical proximity and relationship of the thigh and hip. It is also possible that the previously used term for `thigh' may have developed some negative associations / connotations, so there may be some taboo avoidance in the history of this word.

\addtocounter{subsection}{+1}
\subsection{}

Spanish  {\it ciruela} `plum' $<$ Latin {\it pr\=una c\=ereola} `waxy plum' ({\it pr\=una} `plum' + {\it c\=ereola} `of wax').

This is an example of synecdoche, specifically ellipsis. The former name described a form of the fruit using a noun-adjective phrase. The new name is derived solely from the adjectival part of the name. Possibly because the term {\it pruna} had some other referent. So, to distinguish between the two, the salient portion of the name was the adjectival part.

\addtocounter{subsection}{+1}
\subsection{}

Spanish {\it depender} `to depend' $<$ Latin {\it d\=ependere} `to hang'.

This is an example of metaphor. When an object A is hanging off of another object B, the original construction would be B depends on (possibly off) A. This meaning shifted metaphorically to B has need of A for support. Hence, the current denotation of `depend'.

\section{Exercise 10.3: The development of perfect auxiliaries in Spanish}

{\it Stage I}, as presented in the text, describes the use of the past passive participle in Latin as an adjective, a nominal modifier. As such, it agreed with the associated noun in number, gender, and case, as would other nominal modifiers, such as the colors. This is actually a typical use of the past passive participle. For example, in English `fried eggs', describing eggs which underwent the process of frying. Specifically, the {\it Stage I} write-up describes constructions with the verb {\it hab\=ere} `to have', indicating possession. When used in conjunction with the past passive participle, it signaled aspect, that the action had been completed, that one had in one's possession objects which had completed the process of being X-ed. This is seen in example sentence~(2), {\it habe\=o litter\=as script\=as} `I have letters which are written'.

Some place between {\it Stage I} and {\it Stage II} there has been a re-analysis of sentences like (2) from `I am in possession of letters which were written {\it by someone}.' to `I have letters which I have written.' That is, in the original formulation, the writer of the letter is not specified, whereas under the reanalysis, the writer of the letters becomes tied to the possessor, that is to the subject of the verb {\it haber}.

The {\it Stage II} development is the expansion of the meaning of Old Spanish {\it haber} so that it no longer has the primary meaning of possession, but is being used metaphorically to mark association with an external object. This continues to be used in conjunction with the past passive participle. In the example sentence~(3), {\it Los hab\'ia \ldots hechos enemigos de estotros.} `He had made enemies of these others.', we see an example of this. The primary original meaning of the main verb {\it haber} has been expanded from `{\it agent} is in possession of {\it patient}' to `{\it patient} exists within the world of {\it agent}'. This is now coupled with the agentive association for the past passive participle. However, at this time, the participle is still functioning as a nominal modifier, agreeing with the plural object `enemies', rather than the singular subject `he'. Thus, the postulated paraphrastic gloss for sentence~(3) is `There were in his world enemies whom he had made from the others.' Notice the shift in location of the past passive participle from the typical post-nominal location to pre-nominal, i.e., closer to the implied agent and to the verb.

This sets the stage for a second reanalysis of the construction to the gloss given within the text. In this gloss, the possessive sense of the verb is bleached away. With the loss of this verbal force of {\it haber}, the passive participle is `promoted' to the status of main verb, with the bleached {\it haber} relegated to the role of inflection-bearing auxiliary.

As this shift in verbal assignment occurs, the participle loses its adjectival function. The agreement in number and gender with the patient is also lost. We see this in {\it Stage III}, example~(4) {\it Hemos escrito cartas} `We have written letters.' In this sentence, {\it escrito} bears no markers for gender nor number. Thus a new form of the verb has been added to the paradigm. This new form of the main verb is used with an auxiliary verb {\it haber} which is inflection bearing.

As sentence~(5) shows, the passive participle still is functioning within the language. True to its adjectival role within sentence~(5) {\it Tenemos cartas escritas en tinta roja} `We have letters written in red ink', the participle both appears in the typical post-nominal adjectival position and agrees with the head of the phrase (letters) in number and gender.

In {\it Stage IV}, this aspectual auxiliary (or is it a periphrastic tense marking? \ldots at some level, it is unclear from the write-up and glosses) broadens its application from passive constructions to `intransitive' verbs (those that are unaccusative or unergative) through analogical extension. In the examples listed, the Old Spanish forms use the verb `to be' {\it ser} as the auxiliary with the participle agreeing with the subject in number and gender. In these Old Spanish cases, the participle continued to function as if it were a predicate adjective. In fact, some would probably analyze these sentences as examples of the participle being used as a predicate adjective. The use of {\it haber} then has extended in Modern Spanish from transitive, perfective constructions to intransitive as well. So, the last vestiges of the former verb have been bleached away. And, in so doing, the invariant participial form is used for the main verb. 

It could be noted that this same construction is found in Modern English, with {\it have} being the aspectual auxiliary for both transitive and intransitive verbs. There are a few fossilized usages that belie the older forms, particularly in ecclesiastical usage: `Christ is risen', `The king is come' \ldots.

\end{document}
