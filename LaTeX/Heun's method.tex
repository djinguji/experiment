\documentclass[12pt]{article}

\usepackage{fullpage}
\usepackage{amsmath}
\usepackage{graphics}
%\usepackage{url}
%\usepackage{lingmacros}
%\usepackage{rtrees}
%\usepackage{stmaryrd}

\begin{document}

\renewcommand\thesection {}
\renewcommand\thesubsection {\Alph{subsection}.}

\section{Some Thoughts on Heun's Method}

In the lecture on Monday, Lisa said that she didn't have the rationale for Heun's method immediately at her finger tips. In reviewing the notes, I think I have a simple suggestion.

First, some general terminology. Given an IVP,
\begin{equation*}
\frac{dy}{dt} = f(y(t), t)) \qquad
y(0) = y_0,\;
t \in [t_0, t_N]
\end{equation*}
the general form for the solution is
\begin{equation}
y_{n+1} = y_n + \Delta t\varphi
\end{equation}
where $\varphi$ is some appropriate function of $f$, $y$, and $t$.

In Euler's method, $\varphi$ is the function $f$ from the IVP statement.
That is,
\begin{equation}
y_{n+1} = y_n + \Delta t\cdot f(y(t_n), t_n)
\end{equation}
This can be rewritten,
\begin{equation}
\label{euler}
y(t + \Delta t) = y(t) + \Delta t\cdot f(y(t), t)
\end{equation}
So, (\ref{euler}) is the ``Euler's method approximation for $y(t + \Delta t)$''.

So, in the Modified Euler-Cauchy method,
\begin{equation}
y(t + \Delta t) = y(t) + \Delta t\cdot f\left(y(t) + \frac{\Delta t}{2}f(y(t),t), t + \frac{\Delta t}{2}\right)
\end{equation}
$\varphi$ is the function $f$ evaluated at the midpoint of the interval, $t + \frac{\Delta t}{2}$. It uses the Euler approximation for $y(t + \frac{\Delta t}{2})$, that is $y(t) + \frac{\Delta t}{2}f(y(t),t)$.
\begin{equation}
\varphi = f\left(y(t) + \frac{\Delta t}{2}f(y(t),t), t + \frac{\Delta t}{2}\right)
\end{equation}
So, the name Midpoint method is transparent. 

Similarly, in Heun's method,
\begin{equation}
y(t + \Delta t) = y(t) + \frac{\Delta t}{2}\cdot \left(f(y(t),t) + f(y(t) + \Delta t\cdot f(y(t),t), t + \Delta t)\right)
\end{equation}
$\varphi$ is the mean of the function $f$ evaluate at $t$ and $t + \Delta t$. The value of $f$ at $t$ is straight forward, $f(y(t),t)$. It uses Euler's approximation for $y(t + \Delta t)$, that is $\mathsf{y(t) + \Delta t\cdot f(y(t),t)}$. So the full term for $f$ at $t + \Delta t$ is $f(\mathsf{y(t) + \Delta t\cdot f(y(t),t)}, t + \Delta t)$.
\begin{equation}
\varphi = \frac{f(y(t),t) + f(\mathsf{y(t) + \Delta t\cdot f(y(t),t)}, t + \Delta t)}{2}
\end{equation}
So, Heun's method uses an ``average'' for $\varphi$.

\end{document} 