\documentclass[12pt,USLetter]{article}

\usepackage{fullpage}
\usepackage{amsmath}
\usepackage{amssymb}
\usepackage{graphics}
%\usepackage{fancyhdr}

%\pagestyle{fancy}
%\fancyhf{}
%\lfoot{AMATH 503, Assignment 1}
%\rhead{Dan Jinguji}
%\cfoot{\thepage}

% remove the \Re and \Im operators
%\let \Re \relax
%\let \Im \relax
\DeclareMathOperator \Arg {Arg}
\DeclareMathOperator \Log {Log}
%\DeclareMathOperator \Re {Re}
%\DeclareMathOperator \Im {Im}

\begin{document}

\title{AppMath 503 \\ Homework 1}
\author{Dan Jinguji \\ 7339426}
\date{Due: Monday 14 April 2014}

\maketitle

\renewcommand\thesection {\arabic{section}:}
\renewcommand\thesubsection {(\alph{subsection})}
\newcommand{\vc}[1]{\ensuremath{\,\vec{#1}}}
\newcommand{\pbv}[2]{\ensuremath{\langle{#1}, {#2}\rangle}}
\newcommand{\he}[1]{\ensuremath{\;\hat{e}_{#1}}}
\newcommand{\er}{\he{r}}
\newcommand{\et}{\he{\theta}}
\newcommand{\pd}[1]{\ensuremath{\frac{\partial}{\partial{#1}}}}
\newcommand{\spd}[1]{\ensuremath{\frac{\partial^2}{\partial{#1}^2}}}
\newcommand{\ppd}[2]{\ensuremath{\frac{\partial{#1}}{\partial{#2}}}}
\newcommand{\abs}[1]{\ensuremath{\lvert{#1}\rvert}}
\newcommand{\pr}{\ensuremath{^{\prime}}}
\newcommand{\dpr}{\ensuremath{^{\prime\prime}}}
\newcommand{\Let}{\ensuremath{\mathrm{Let\,}}}
\newcommand{\atan}{\ensuremath{\mathrm{atan2\,}}}
\newcommand{\where}{\ensuremath{\mathrm{where\,}}}
\newcommand{\signtoinf}{\ensuremath{\Sigma_{n=1}^{\infty}}}
\newcommand{\half}{\ensuremath{\tfrac{1}{2}}}

\section{}

The cable equation:
\begin{equation*}
\pd{t}v = \gamma\spd{x}v - \alpha v, \qquad \mathrm{with\,} \alpha,\gamma > 0
\end{equation*}
also known as the lossy heat equation, was derived by the nineteenth-century Scottish physicist William Thomson to model propagation of signals in transatlantic cables.

\subsection{}

Show that the general solution for this equation is given by: $v(x,t) = e^{-\alpha t} u(x,t)$, where $u(x,t)$ solves the heat equation.

\begin{align*}
\pd{t}u(x,t) &= \gamma \spd{x}u(x,t) \qquad \mathrm{given} \,
v(x,t) = e^{-\alpha t} u(x,t) \\
\spd{x}v(x,t) &= e^{-\alpha t} \spd{x}u(x,t) \\
\pd{t}v &= (-\alpha) e^{-\alpha t} u(x,t) + e^{-\alpha t} \pd{t}u(x,t) \\
&= -\alpha e^{-\alpha t} u(x,t) + e^{-\alpha t} \gamma \spd{x}u(x,t) \\
&= -\alpha v(x,t) + \gamma \spd{x}v(x,t) \\
&= \gamma \spd{x}v(x,t) -\alpha v(x,t) \qquad \mathrm{QED}
\end{align*}

\subsection{}

Find the Fourier series solution to this equation subject to:
\begin{equation*}
v(0,t) = 0 = v(1,t), v(x,0) = f(x), 0 \le x \le 1, t > 0.
\end{equation*}
Does your solution approach equilibrium? How fast? \\

If we take our general solution from part (a), $v(x,t) = e^{-\alpha t} u(x,t)$. We know we can solve the heat equation using separation of variables:
\begin{align*}
u(x,t) &= X(x)T(t) \\
\pd{t}u(x,t) &= \gamma \spd{x}u(x,t) \\
XT^{\prime} &= \gamma X^{\prime\prime} T \\
\frac{T^{\prime}}{\gamma T} &= \frac{X^{\prime\prime}}{X} = -\lambda^2 \\
X\dpr &= -\lambda^2 X \\
X(x) &= A \sin(\lambda x) + B \cos(\lambda x) \\
X(0) &= 0 \qquad \because v(0,t) = 0 \\
&= A \sin(0) + B \cos(0) \qquad \therefore B = 0 \\
X(x) &= A\sin(\lambda x) \\
X(1) &= 0 \qquad \because v(1,t) = 0 \\
&= A\sin(\lambda) \qquad \therefore \lambda = n\pi \\
X(x) &= A_n\sin(n\pi x) \\
\frac{T^{\prime}}{\gamma T} &= -\lambda^2 \\
T\pr &= - \gamma (n\pi)^2 T \\
T(t) &= T_n(0)e^{-\gamma (n\pi)^2 t} \\
v(x,t) &= e^{-\alpha t} A_n\sin(n\pi x) T_n(0)e^{-\gamma(n\pi)^2 t} \\
v(x,0) &= f(x) = A_n\sin(n\pi x) T_n(0) \\
&= C_n\sin(n\pi x) \\
\int_0^1 f(x)\sin(m\pi x) dx &= \int_0^1 C_n \sin(m\pi x)\sin(n\pi x) dx \\
\end{align*}

Our function $v(x,t)$ is well behaved given these BCs and ICs. It is dominated by the $e{-\alpha t}$ term, so will reach equilibrium faster given larger values of $\alpha$, $ \alpha > 0$ per the problem statement.

\subsection{}

Redo part (b) but with the boundary condition:
\begin{equation*}
\pd{x}v(0,t) = 0 = \pd{x}v (1, t)
\end{equation*}

This problem is ill-posed. Since the boundary conditions are both Neumann boundary conditions it does not sufficiently specify the behavior of the solution.

\section{}

Find the solution to the following wave problems in the form of a Fourier series:

\subsection{}

\begin{equation*}
\spd{t}u = \spd{x}u, u(0,t) = u(\pi, t) = 0, u(x,0) = 1, \pd{t}u(x,0) = 0, 0<x<\pi
\end{equation*}

Assume separable function $u(x,t) = X(x)T(t)$.

\begin{align*}
XT\dpr &= X\dpr T \\
\frac{T\dpr}{T} &= \frac{X\dpr}{X} = -\lambda^2 \\
X(x) &= A\sin(\lambda x) + B\cos(\lambda x) \\
X(0) &= 0 \qquad \because u(0,t) = 0 \\
&= A\sin(0) + B\cos(0) \qquad \therefore B = 0 \\
X(x) &= A\sin(\lambda x) \\
X(\pi) &= 0 \qquad \because u(\pi,t) = 0 \\
&= A\sin(\lambda \pi) = 0 \qquad \therefore \lambda \in J \\
X(x) &= A_n\sin(nx) \\
T(t) &= \alpha\sin(nt) + \beta\cos(nt) \\
u(x,t) &= \Sigma_{n=1}^{\infty}\sin(nx) (\alpha_n\sin(nt) + \beta_n\cos(nt)) \\
\pd{t}u(x,t) &= \Sigma_{n=1}^\infty\sin(nx) (\alpha_n n\cos(nt) - \beta_n n\sin(nt)) \\
\pd{t}u(x,0) &= 0 = \Sigma_{n=1}^\infty\sin(nx) (\alpha_n n\cos(0) - \beta_n n\sin(0)) \qquad \therefore \alpha_n = 0
\end{align*}
\begin{align*}
u(x,t) &= \Sigma_{n=1}^{\infty}\sin(nx) (\beta_n\cos(nt)) \\
u(x,0) &= 1 = \signtoinf\sin(nx) (\beta_n\cos(0)) \\
1 &= \signtoinf\sin(nx)\beta_n \\
\int_0^\pi\sin(mx)dx &= \beta_n\int_0^\pi\sin(mx)\sin(nx)dx \\
\beta_n &= 2 / n \;\where n \in {1, 3, 5, 7, 9 ...} \\
u(x,t) &= \signtoinf\sin((2n-1)x) (2/(2n-1)\cos((2n-1)t))
\end{align*}

\subsection{}

\begin{equation*}
\spd{t}u = 2\spd{x}u, u(0,t) = u(\pi, t) = 0, u(x,0) = 0, \pd{t}u(x,0) = 1, 0<x<\pi
\end{equation*}

Assume separable function $u(x,t) = X(x)T(t)$.

\begin{align*}
XT\dpr &= 2X\dpr T \\
\frac{T\dpr}{T} &= \frac{2X\dpr}{X} = -\lambda^2 \\
X(x) &= A\sin(\surd 2\lambda x) + B\cos(\surd 2\lambda x) \\
X(0) &= 0 \qquad \because u(0,t) = 0 \\
&= A\sin(0) + B\cos(0) \qquad \therefore B = 0 \\
X(x) &= A\sin(\surd 2\lambda x) \\
X(\pi) &= 0 \qquad \because u(\pi,t) = 0 \\
&= A\sin(\surd 2\lambda\pi) \qquad \therefore \lambda \in \{\surd 2/2, 2\surd 2/2, 3\surd 2/2, \dots \} \\
\end{align*}

\subsection{}

\begin{equation*}
\spd{t}u = 3\spd{x}u, u(0,t) = u(\pi, t) = 0, u(x,0) = \sin^3 x, \pd{t}u(x,0) = 0, 0<x<\pi
\end{equation*}

Assume separable function $u(x,t) = X(x)T(t)$.

\begin{align*}
XT\dpr &= 3X\dpr T \\
\frac{T\dpr}{T} &= \frac{3X\dpr}{X} = -\lambda^2 \\
X(x) &= A\sin(\surd 3\lambda x) + B\cos(\surd 3\lambda x) \\
X(0) &= 0 \qquad \because u(0,t) = 0 \\
0 &= A\sin(\surd 3\lambda 0) + B\cos(\surd 3\lambda 0) \quad \therefore B = 0 \\
\end{align*}

\subsection{}

\begin{equation*}
\spd{t}u = 4\spd{x}u, u(0,t) = u(\pi, t) = 0, u(x,0) = x, \pd{t}u(x,0) = -x, 0<x<\pi
\end{equation*}

Assume separable function $u(x,t) = X(x)T(t)$.

\begin{align*}
XT\dpr &= 4X\dpr T \\
\frac{T\dpr}{T} &= \frac{4X\dpr}{X} = -\lambda^2 \\
X(x) &= A\sin(2\lambda x) + B\cos(2\lambda x) \\
X(0) &= 0 \qquad \because u(0,t) = 0 \\
&= A\sin(0) + B\cos(0) \qquad \therefore B = 0 \\
X(x) &= A\sin(2\lambda x) \\
X(\pi) &= 0 \qquad \because u(\pi,t) = 0 \\
&= A\sin(2\lambda\pi) \qquad \therefore \lambda = n/2 \\
&= A_n\sin(\tfrac{1}{2}nx) \\
T(t) &= \alpha\sin(\lambda t) + \beta\cos(\lambda t) \\
&= \alpha\sin(\tfrac{1}{2}nt) + \beta\cos(\tfrac{1}{2}nt) \\
u(x,t) &= \signtoinf \sin(\tfrac{1}{2}nx)(\alpha_n\sin(\half nt) + \beta_n\cos(\half nt)) \\
u(x,0) &= \signtoinf \sin(\tfrac{1}{2}nx)(\alpha_n\sin(0) + \beta_n\cos(0)) \\
x &= \signtoinf \sin(\half nx) \beta_n
\end{align*}
\begin{align*}
\int_0^\pi x\sin(\half mx) dx &= \signtoinf \beta_n \int_0^\pi \sin(\half mx)\sin(\half nx) dx \\
\tfrac{4}{m^2}\sin(\half \pi m) - \tfrac{2\pi}{m}\cos(\half \pi m)
&= \beta_m (\half\pi - \tfrac{1}{2m}\sin(\pi m)) \\
\pd{t}u(x,t) &= \signtoinf \sin(\half nx) (\half\alpha_nn\cos(\half nt) - \half\beta_nn\sin(\half nt)) \\
\pd{t}u(x,0) &= -x = \signtoinf\sin(\half nx) \half(\alpha_nn\cos(0) - \beta_nn\sin(0)) \quad \therefore  \\
\end{align*}

\subsection{}

\begin{equation*}
\spd{t}u = \spd{x}u, u(0,t) = 0, \pd{x}u(\pi, t) = 0, u(x,0) = 1, \pd{t}u(x,0) = 0, 0<x<\pi
\end{equation*}

Assume separable function $u(x,t) = X(x)T(t)$.

\begin{align*}
XT\dpr &= X\dpr T \\
\frac{T\dpr}{T} &= \frac{X\dpr}{X} = -\lambda^2 \\
X(x) &= A\sin(\lambda x) + B\cos(\lambda x) \\
X(0) &= 0 \qquad \because u(0,t) = 0
\end{align*}

\subsection{}

\begin{equation*}
\spd{t}u = 2\spd{x}u, \pd{x}u(0,t) = \pd{x}u(2\pi,t) = 0, u(x,0) = -1, \pd{t}u(x, 0) = 1, 0<x<2\pi
\end{equation*}

Assume separable function $u(x,t) = X(x)T(t)$.

\begin{align*}
XT\dpr &= 2X\dpr T \\
\frac{T\dpr}{T} &= \frac{2X\dpr}{X} = -\lambda^2 \\
X(x) &= A\sin(\lambda x) + B\cos(\lambda x) \\
X(0) &= 0 \qquad \because u(0,t) = 0
\end{align*}

\subsection{}

\begin{equation*}
\spd{t}u = \spd{x}u, \pd{x}u(0,t) = \pd{x}u(1,t) = 0, u(x,0) = x(1-x), \pd{t}u(x, 0) = 0, 0<x<1
\end{equation*}

Assume separable function $u(x,t) = X(x)T(t)$.

\begin{align*}
XT\dpr &= X\dpr T \\
\frac{T\dpr}{T} &= \frac{X\dpr}{X} = -\lambda^2 \\
X(x) &= A\sin(\lambda x) + B\cos(\lambda x) \\
X\pr &= A\lambda\cos(\lambda x) - B\lambda\sin(\lambda x) \\
X\pr(0) &= 0 \qquad \because \pd{x}u(0,t) = 0 \; \therefore A=0 \\
\end{align*}

\end{document}
